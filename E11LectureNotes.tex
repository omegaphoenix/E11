\documentclass[10pt]{article}
\usepackage{amssymb}
\usepackage{amsmath}
\usepackage{amsthm}
\usepackage{amsfonts}
\usepackage{algorithm}
\usepackage[noend]{algpseudocode}
\usepackage{enumerate}
\usepackage{hyperref}
\usepackage{listings}
\usepackage{enumitem}
\usepackage[margin=0.3in]{geometry}
\title{\textbf{Introduction to Science and Engineering Genres}}
\author{Justin Leong}
\date{\today}
\begin{document}

\maketitle
\setlength\parindent{0pt}
\setlist[description]{style=nextline,leftmargin=0cm}

\begin{description}
\section{Session 1 - 4/5/17}
\item[Who is Christina Birch?]
  really likes getting people to do things by communicating effectively (usually to give her money);
  loves science and engineering and cycling;
  likes communicating;
  Ph.D. in bio engineering in MIT
\item[What are the course objectives?]
  conduct literature searches;
  identify audience and tailor to them;
  understand the anatomy of different scientific papers;
  review scientific writing and providing feedback to academic peers;
  practice professional communication (email, cover letters, rebuttals, responses)
\item[What is the course structure?]
  select research project or topic and write original journal article;
  identify a target science or engineering journal or conference to which paper could be submitted;
  partner with a professor, research staff, postdoc, or senior graduate student to serve as a content expert to review your paper
\item[Read first chapter of Made to Stick]
\item[What are the different types of scientific writing genres?]
  science fiction, research papers, code documentation, poster presentation, research proposal, grant request;
  textbook, patent, news article, press release, specs
\item[Who is the audience of these genres?]
  grant - academic scientists and engineers, generals (military), government, HR
\item[How does language differ between between different journals?]
  Scientific American - ``reflecting bast progress, hardest-hit, dramatically'' - sexy science;
  government report - more structure (form important, want to make it skimable), ``vector control''; brief, concise, evidence focused, specifis details but briefly, consistent phrase (vector control);
  scientific paper - stating data while trying to persuade you it is something worthwhile (more money in the future; human progress
\item[What is the purpose of a scientific paper?]
  persuade people to do something;
  summarize, argue, persuade, narrate ,guide, evaluate, analyze, respond, examine, investigate;
  influence the way people think;
  present a coherent argument (engineering writing must be trustworthy and persuasive, writing and research go hand-in-hand, logical project plan is easy to understand, credible)
\item[What is the scientific method?]
  Make an observation, pose a question, propose a hypothesis, design an experiment, execute and gather data, analyze, draw conclusions;
  purpose, background research, experimentation and observations, analysis, conclusions
\item[what is the strcuture of a research article?]
  motivation for the work, any background;
  problem or knowledge gap;
  propose a solution or explanation;
  methods and results;
  analysis of data;
  conclusions and broader significance
\item[what is the strcuture of a review article?]
  Motivation for filed of study, background necessary;
  recent problems
\item[what is the strcuture of a design report?]
  purpose, introduction;
  problem definition, design description, design evaluation;
  recommend action based on evaluated results, propose more studies, conclusions
\item[what is the strcuture of a proposal?]
  purpose, introduction;
  problem definition, \dots
  expected result, \dots
\item[How do you support your claim?]
  context, justificaiton, evidence, constraint, counterclaim
\item[Assignments]
  Get on Moodle;
  talk to at least two potential content experts;
  Read Chapter 1 of Made to Stick
\end{description}

\begin{description}
\section{Session 2 - 4/12/17 - Planning a scientific narrative - Writing heuristics to create maps for manuscripts}
\item[Homework]
  1. Read Elements of Style article.
  2. Write a brief, one-paragraph proposal on your topic, followed by a ~2 page outline of your paper. Bring a printed copy to class.
  3. Submit your topic selection and content expert form online.
\item[What do you need to know when you write a paper? Key Elements for Successful Communication]
  Why - strategic purpose (commander's intent);
  Who - audience;
  What - context, ideas;
  How - organizational structure, flow, visual impact, synergy;
\item[What is the anatomy of a manuscrift?]
  Motivation for the work. Background.
  Problem of knowledge gap. Solution. Methods and Results.
  Analysis of data. Conclusions and broader significance.
\item[Anatomy of a manuscript]
  Hourglass shape - whole paper and on paragraph levels.
  Background information and motivation.
  Specific research question or knowledge gap.
  Your solution.
  Major findings or results.
  Significance and broader implications of your work.
\item[Prewriting Step 1]
  Brainstorm important concepts, sketch out major ideas.
  Just write! Cubing. Organize later.
  Cubing - describe it. Compare it. Associate it. Argue for/against it. Analyze it. Apply it.
\item[Prewriting Step 2]
  Group major concepts
\item[Prewriting Step 3]
  Organize around key figures. Try storyboarding.
\item[Prewriting Step 4]
  Fill in outline by writing sentence
\item[Prewriting Step 5]
  Get feedback from peers.
\subsection{Library Resources and Citation Management using EndNote}
\item[What is research?]
  You serarch. And you search again.
\item[What are the steps?]
  Background/Lit Search;
  Experiments;
  Communication Results;
\item[What do you need in your background and literature search?]
  Summarize past work;
  Highlight themes;
  Note where holes are;
  Establish novelty. Establish credibility. Establish authority.
\item[What are databases to use?]
  Web of Science; Google Scholar (not really a database)
\end{description}


\begin{description}
\section{Session 3 - 4/19/17 - Setting the stage}
\item[Elements of Style]
  Conclusions add something new;
  content more important than structure;
  remember to use commander's intent throughout;
  concrete - avoid rabbit holes;
  terminology - avoiding jargon;
  resolutions and next steps;
  INTERESTING;
  Story is the concept that should underlie the structure of the entire paper. The clearer and simpler, the more engrossing it is.
  You might want to offer specific problems that could be addreessed or new capaiblities that are enabled by your work.
\item[What makes a compelling narrative?]
  Motivation is clear (early  on, background?);
  Problem and solution (logic, thought process, path);
  1st paragraph;
  easy to comprehend figures "don't say too much";
\item[Anatomy of a Manuscript - Review Paper]
  middle of hourclass - significant breakthrough
\item[What is the purpose of an introduction? What information should you include to accomplish this?]
  motivation, goal, enough context, trying to convince reader to read rest of paper;
  (title, abstract, and first paragraph);
  You have only the first two or three sentences to catch the reader's attention.
  Don't waste them on generalities.
  Preview the paper (documents  as mapped fractals):
    Why you/they did it: Motivation;
    What you/they did: Principal result;
    Who cares: Importance
\item[Introductions provide (only) the context a reader needs]
  Provide context to orient readers who are less familiar with your topic and to establish the specific importance of your work.
  State the need for your work as an opposition between what the scientfic community currently has and what it wants.
  Indicate what solution is employed to address the need (discuss a high-level approach).
  Preview the remainder of the paper to prepare readers for its structure.
\item[Why is the first sentence of the dog paper good?]
  It uses the word "suck" and it is easy to read.
  Word choice is easy to read.
  Introduction - General background, specific background, need, solution
\end{description}


\begin{description}
\section{Session 4 - 4/26/17}
\item[What is wrong with the tradtitional IMRAD structure?]
    Might make sense to talk about method 1.0, 2.0, etc. with case studies;
    Don't force content into holes made by existing section headers.
    Descriptive Section Headers are more effective
\item[Network Analysis Facebook paper]
    Central Question - specific;
    Methods - Logos (logical step by step);
    Approach;
    Method and Design;
    first person, present tense active - real time, exciting, concise;
    set up/motivation to result and use context to bring up claims and question;
    existing theory to current problem (explaining  first pass approach and why we aren't using it)  - Reminder: liimits/motivation to Results: Theory;
\item[Descriptive section headers]
    descriptive section headers correspond to figures (is there a figure supplementing each section?);
    mixes simulation with theory based on which is more effective
\item[Data/Ink(pixel)]
    maximize data by minimizing ink and pixels while keeping message intact;
    don't make figures 3d;
    don't need legend (just label directly);
    bigger text
\end{description}
\end{document}
