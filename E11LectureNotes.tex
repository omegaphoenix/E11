\documentclass[10pt]{article}
\usepackage{amssymb}
\usepackage{amsmath}
\usepackage{amsthm}
\usepackage{amsfonts}
\usepackage{algorithm}
\usepackage[noend]{algpseudocode}
\usepackage{enumerate}
\usepackage{hyperref}
\usepackage{listings}
\usepackage{enumitem}
\usepackage[margin=0.3in]{geometry}
\title{\textbf{Introduction to Science and Engineering Genres}}
\author{Justin Leong}
\date{\today}
\begin{document}

\maketitle
\setlength\parindent{0pt}
\setlist[description]{style=nextline,leftmargin=0cm}

\begin{description}
\section{Session 1 - 4/5/17}
\item[Who is Christina Birch?]
  really likes getting people to do things by communicating effectively (usually to give her money);
  loves science and engineering and cycling;
  likes communicating;
  Ph.D. in bio engineering in MIT
\item[What are the course objectives?]
  conduct literature searches;
  identify audience and tailor to them;
  understand the anatomy of different scientific papers;
  review scientific writing and providing feedback to academic peers;
  practice professional communication (email, cover letters, rebuttals, responses)
\item[What is the course structure?]
  select research project or topic and write original journal article;
  identify a target science or engineering journal or conference to which paper could be submitted;
  partner with a professor, research staff, postdoc, or senior graduate student to serve as a content expert to review your paper
\item[Read first chapter of Made to Stick]
\item[What are the different types of scientific writing genres?]
  science fiction, research papers, code documentation, poster presentation, research proposal, grant request;
  textbook, patent, news article, press release, specs
\item[Who is the audience of these genres?]
  grant - academic scientists and engineers, generals (military), government, HR
\item[How does language differ between between different journals?]
  Scientific American - ``reflecting bast progress, hardest-hit, dramatically'' - sexy science;
  government report - more structure (form important, want to make it skimable), ``vector control''; brief, concise, evidence focused, specifis details but briefly, consistent phrase (vector control);
  scientific paper - stating data while trying to persuade you it is something worthwhile (more money in the future; human progress
\item[What is the purpose of a scientific paper?]
  persuade people to do something;
  summarize, argue, persuade, narrate ,guide, evaluate, analyze, respond, examine, investigate;
  influence the way people think;
  present a coherent argument (engineering writing must be trustworthy and persuasive, writing and research go hand-in-hand, logical project plan is easy to understand, credible)
\item[What is the scientific method?]
  Make an observation, pose a question, propose a hypothesis, design an experiment, execute and gather data, analyze, draw conclusions;
  purpose, background research, experimentation and observations, analysis, conclusions
\item[what is the strcuture of a research article?]
  motivation for the work, any background;
  problem or knowledge gap;
  propose a solution or explanation;
  methods and results;
  analysis of data;
  conclusions and broader significance
\item[what is the strcuture of a review article?]
  Motivation for filed of study, background necessary;
  recent problems
\item[what is the strcuture of a design report?]
  purpose, introduction;
  problem definition, design description, design evaluation;
  recommend action based on evaluated results, propose more studies, conclusions
\item[what is the strcuture of a proposal?]
  purpose, introduction;
  problem definition, \dots
  expected result, \dots
\item[How do you support your claim?]
  context, justificaiton, evidence, constraint, counterclaim
\item[Assignments]
  Get on Moodle;
  talk to at least two potential content experts;
  Read Chapter 1 of Made to Stick
\end{description}

\begin{description}
\section{Session 2 - 4/12/17 - Planning a scientific narrative - Writing heuristics to create maps for manuscripts}
\item[Homework]
  1. Read Elements of Style article.
  2. Write a brief, one-paragraph proposal on your topic, followed by a ~2 page outline of your paper. Bring a printed copy to class.
  3. Submit your topic selectio nand content expert form online.
\item[What do you need to know when you write a paper? Key Elements for Successful Communication]
  Why - strategic purpose (commander's intent);
  Who - audience;
  What - context, ideas;
  How - organizational structure, flow, visual impact, synergy;
\item[What is the anatomy of a manuscrift?]
  Motivation for the work. Background.
  Problem of knowledge gap. Solution. Methods and Results.
  Analysis of data. Conclusions and broader significance.
\item[Anatomy of a manuscript]
  Hourglass shape - whole paper and on paragraph levels.
  Background information and motivation.
  Specific research question or knowledge gap.
  Your solution.
  Major findings or results.
  Significance and broader implications of your work.
\item[Prewriting Step 1]
  Brainstorm important concepts, sketch out major ideas.
  Just write! Cubing. Organize later.
  Cubing - describe it. Compare it. Associate it. Argue for/against it. Analyze it. Apply it.
\item[Prewriting Step 2]
  Group majjor concepts
\item[Prewriting Step 3]
  Organize around key figures. Try storyboarding.
\item[Prewriting Step 4]
  Fill in outline by writing sentence
\item[Prewriting Step 5]
  Get feedback from peers.
\subsection{Library Resources and Citation Management using EndNote}
\item[What is research?]
  You serarch. And you search again.
\item[What are the steps?]
  Background/Lit Search;
  Experiments;
  Communication Results;
\item[What do you need in your background and literature search?]
  Summarize past work;
  Highlight themes;
  Note where holes are;
  Establish novelty. Establish credibility. Establish authority.
\item[What are databases to use?]
  Web of Science; Google Scholar (not really a database)
\end{description}

\end{document}
